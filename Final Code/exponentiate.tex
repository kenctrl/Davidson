% This file was automatically created from the m-file 
% "m2tex.m" written by USL. 
% The fontencoding in this file is UTF-8. 
%  
% You will need to include the following two packages in 
% your LaTeX-Main-File. 
%  
% \usepackage{color} 
% \usepackage{fancyvrb} 
%  
% It is advised to use the following option for Inputenc 
% \usepackage[utf8]{inputenc} 
%  
  
% definition of matlab colors: 
\definecolor{mblue}{rgb}{0,0,1} 
\definecolor{mgreen}{rgb}{0.13333,0.5451,0.13333} 
\definecolor{mred}{rgb}{0.62745,0.12549,0.94118} 
\definecolor{mgrey}{rgb}{0.5,0.5,0.5} 
\definecolor{mdarkgrey}{rgb}{0.25,0.25,0.25} 
  
\DefineShortVerb[fontfamily=courier,fontseries=m]{\$} 
\DefineShortVerb[fontfamily=courier,fontseries=b]{\#} 
  
\noindent                       
 \hspace*{-1.6em}{\scriptsize 1}$  $\color{mblue}$function$\color{black}$ v_evolved = exponentiate(v_in,H_in,dt_in,subdivisions)$\\
 \hspace*{-1.6em}{\scriptsize 2}$  $\color{mgreen}#%% EXPONENTIATE performs time evolution on an inpute state using a Taylor#\color{black}$$\\
 \hspace*{-1.6em}{\scriptsize 3}$  $\color{mgreen}#%% approximation of the input Hamiltonian.#\color{black}$$\\
 \hspace*{-1.6em}{\scriptsize 4}$      $\color{mgreen}$% v_in is the input state.$\color{black}$$\\
 \hspace*{-1.6em}{\scriptsize 5}$      $\color{mgreen}$% H_in is the input Hamiltonian.$\color{black}$$\\
 \hspace*{-1.6em}{\scriptsize 6}$      $\color{mgreen}$% dt is the time step for which time evolution occurs. dt_in is defined$\color{black}$$\\
 \hspace*{-1.6em}{\scriptsize 7}$      $\color{mgreen}$% as dt multiplied by a factor of -i when using EXPONENTIATE.$\color{black}$$\\
 \hspace*{-1.6em}{\scriptsize 8}$      $\color{mgreen}$% subdivisions is the number of times the Taylor approximation is$\color{black}$$\\
 \hspace*{-1.6em}{\scriptsize 9}$      $\color{mgreen}$% used in the time step.$\color{black}$$\\
 \hspace*{-2em}{\scriptsize 10}$  $\\
 \hspace*{-2em}{\scriptsize 11}$  $\color{mgreen}#%% Split the time step based on the number of subdivisions.#\color{black}$$\\
 \hspace*{-2em}{\scriptsize 12}$  ddt = dt_in/subdivisions;$\\
 \hspace*{-2em}{\scriptsize 13}$  $\\
 \hspace*{-2em}{\scriptsize 14}$  $\color{mgreen}#%% Define v_evolved to be equal to the time evolution of v_in for the first#\color{black}$$\\
 \hspace*{-2em}{\scriptsize 15}$  $\color{mgreen}#%% time subdivision.#\color{black}$$\\
 \hspace*{-2em}{\scriptsize 16}$  v_evolved = v_in + ddt*H_in*v_in + ddt^2/2*H_in*(H_in*v_in)$\color{mblue}$ ...$\color{black}$$\\
 \hspace*{-2em}{\scriptsize 17}$      + ddt^3/6*H_in*(H_in*(H_in*v_in));$\\
 \hspace*{-2em}{\scriptsize 18}$  $\\
 \hspace*{-2em}{\scriptsize 19}$  $\color{mgreen}#%% Continue evolving v_evolved for each time subdivision in dt.#\color{black}$$\\
 \hspace*{-2em}{\scriptsize 20}$  $\color{mblue}$for$\color{black}$ nt = 1:subdivisions-1$\\
 \hspace*{-2em}{\scriptsize 21}$      v_evolved = v_evolved + ddt*H_in*v_evolved + ddt^2/2*H_in$\color{mblue}$ ...$\color{black}$$\\
 \hspace*{-2em}{\scriptsize 22}$          *(H_in*v_evolved) + ddt^3/6*H_in*(H_in*(H_in*v_evolved));$\\
 \hspace*{-2em}{\scriptsize 23}$  $\color{mblue}$end$\color{black}$$\\ 
  
\UndefineShortVerb{\$} 
\UndefineShortVerb{\#}